Étant moi-même concerné par le TDAH (Trouble de l'attention avec/sans hyperactivité), je comprends les défis uniques que cela peut poser dans le monde professionnel. Ce travail vise à démystifier les complexités et les nuances liées à cette condition, souvent incomprise. J'aborderai non seulement les difficultés, mais aussi les talents cachés que le TDAH peut débloquer dans un cadre de travail.
\\
\\
La médication, souvent sujet de controverses et de stigmates, sera également explorée. Cette question, fréquemment gérée par des psychiatres, est délicate. Mon objectif à travers ce travail est de proposer une analyse à la fois équilibrée et approfondie, soutenue par des données factuelles et mon expérience 
\\
\\
Ce travail a également pour ambition d'ouvrir la discussion sur une problématique qui affecte beaucoup d'entre nous mais demeure souvent méconnue dans le milieu professionnel.
\\
\\
Bien sûr, ce travail vise non seulement à éclaircir les défis associés au TDAH, mais également à démanteler les stéréotypes qui l'entourent. Vous pourriez réaliser que des caractéristiques généralement vues comme des défauts peuvent effectivement servir d'atouts dans certaines situations.
\\
\\
Nous aborderons également la complexe problématique de révéler votre condition à vos supérieurs et collègues, en considérant des variables comme la culture organisationnelle et votre confort personnel. Faut-il le faire ou pas ? Et si oui, comment ? La réponse n'est pas simple et dépend de nombreux facteurs, notamment la culture de l'entreprise et votre propre niveau de confort.
\\
\\
En somme, ce travail aspire à être une ressource complète, destinée à aider ceux qui vivent avec le TDAH à naviguer dans le monde professionnel. Il s'agit non seulement de survie, mais de véritable épanouissement.
\\
\\
Je travaillerai sur ce TPA essentiellement les weekends et en cours directement, n'ayant pas vraiment le temps pour ça au travail. La problématique étant connue, je vais pouvoir effectuer mes recherches sur internet directement. Je réaliserai le document sur mon ordinateur en utilisant \href{https://www.latex-project.org/}{LaTeX}.